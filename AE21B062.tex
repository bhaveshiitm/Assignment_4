
\documentclass[12pt,a4paper]{article} \usepackage{amsmath} \usepackage{amsfonts} \usepackage{amssymb}
\usepackage{graphicx} \usepackage[left=2cm,right=2cm,top=2cm,bottom=2cm]{geometry} 
\author{Bhavesh Tongaria} \title{Assignment 4} 
\date{July 07, 2022} 
\begin{document} 
\section{AE21B062}


\maketitle{Relative velocity} 

\section{Introduction} 
Every motion in this universe occurs relative to each other, this concept can help us to understand the velocities in different situtions. 
\section{Discussion} 
The relative velocity is known as the velocity of a on ground with respect to b on ground means the velocity of a when b is
stationary. This helps us a lot in situations where the particles move in translation or rotate with respect to each other. We can see
the examples of this in our day to day life, like in trains and cars etc. These equations are even helpful to solve motions of rigid rods
in engines and machines, like slider crenck mechanism 
\begin{equation} 
\vec{v_a} = \vec{v_r} + \vec{v_b} + \vec{\omega_r}\times\vec{R}
\end{equation} 

This is the base equation that can explation the relative translational and rotational motion between any two bodies. 
\begin{itemize} 
\item eg1:- Relative motion between two cars. 
\item eg2:- Relative motion betwen engine piston and shafts etc. 
\end{itemize} 
\section{Table} 
\begin{table} 
\begin{center} 
	\begin{tabular}{|c|r|l|}
 \hline
	 & symbol & meaning \\
\hline
	1 & $\vec{v_r}$ & The relative velocity of ‘a’ with respect to ‘b’\\
        2 & $\vec{v_a}$ & Velocity of a with respect to ground \\
        3 & $\vec{v_b}$ & Velocity of b with respect to ground \\
	4 & $\vec{\omega_r}$ & Angular velocity of ‘a’ with respec to ‘b’ \\
 	5 & $\vec{R}$ & displacement vector $\vec{ab}$ \\ 
\hline
	\end{tabular} 
	\caption{Equation symbols with their meanings} 
\end{center} 
\end{table} 
\section{conclusion} As per the data provided below we can get to know about relative motion, it’s equations and the day to day
applications of this concept. This may help us to undestand the world around us in a more practical way. 
\bibliography{refs}
\bibliographystyle{alpha} 
\end{document}
